\documentclass[11pt]{beamer}
\usepackage{listings} % Include the listings-package
\usepackage[T1]{fontenc}
\usepackage[utf8]{inputenc}
\usepackage[english]{babel}
\usepackage{amsmath}
\usepackage{amssymb, amsfonts, latexsym, cancel}
\usepackage{float}
\usepackage{graphicx}
\usepackage{epstopdf}
\usepackage{subfigure}
\usepackage{hyperref}
%\usepackage{authblk}
\usepackage{blindtext}
\usepackage{booktabs} % Allows the use of \toprule, 
\usepackage{filecontents}
\usepackage{courier} %% Sets font for listing as Courier.
\usepackage{listings}
%\usepackage{listings, xcolor}
\lstset{
tabsize = 2, %% set tab space width
showstringspaces = false, %% prevent space marking in strings, string is defined as the text that is generally printed directly to the console
numbers = left, %% display line numbers on the left
commentstyle = \color{green}, %% set comment color
keywordstyle = \color{blue}, %% set keyword color
stringstyle = \color{red}, %% set string color
rulecolor = \color{black}, %% set frame color to avoid being affected by text color
basicstyle = \small \ttfamily , %% set listing font and size
breaklines = true, %% enable line breaking
numberstyle = \tiny,
}
\usepackage{caption}
\DeclareCaptionFont{white}{\color{white}}
\DeclareCaptionFormat{listing}{\colorbox{gray}{\parbox{\textwidth}{#1#2#3}}}
\captionsetup[lstlisting]{format=listing,labelfont=white,textfont=white}
\definecolor{urlColor}{rgb}{0.06, 0.3, 0.57}
\definecolor{linkColor}{rgb}{0.57, 0.0, 0.04}
\definecolor{fileColor}{rgb}{0.0, 0.26, 0.26}
\hypersetup{
    colorlinks=true,
    linkcolor=linkColor,
    filecolor=fileColor,      
    urlcolor=urlColor,
}
\urlstyle{same}
\setbeamercovered{transparent}
%\usetheme{Boadilla}
\usetheme{CambridgeUS}
%\usetheme{Berkeley}
%\usetheme{Warsaw}
%\usetheme{Madrid}

\title[Presentación]{\bf\Huge Técnicas de Gamificación y mejoras de la UX}


\author[Group-3]
{
	Jimy Gabriel Revilla Tellez \\
	Carlos Bryan Gallegos Batallanos \\
	Rodrigo Sebastián Huaman Maqque \\
	Jordy Pedro Valencia Jara 
}
\institute[UNSA]
{
% 
System Engineering School\\
System Engineering and Informatic Department\\
Production and Services Faculty\\
San Agustin National University of Arequipa
}
%\logo{\includegraphics[width=3.0cm]{img/logo_unsa.jpg}}


\begin{document}


\begin{frame}
\titlepage
\end{frame}


\begin{frame}
\frametitle{Content}
\tableofcontents
\end{frame}


\section{Calidad de Sistemas Interactivos}
\begin{frame}
\frametitle{Calidad de Sistemas Interactivos}
\begin{itemize}
    \item Podemos medirlo desde diferentes principios
\end{itemize}
\begin{figure}[h!]
    \includegraphics[width=12.0cm]{mapaUtilidad.jpg}
    \label{fig:my_label}
\end{figure}
\end{frame}


\section{Experiencia de Usuario}
\begin{frame}
\frametitle{Diseño emocional}
\begin{itemize}
    \item Debemos buscar establecer una relacion emocional con los objetos
\end{itemize}
\begin{figure}
    \centering
    \includegraphics[width=8.0cm]{imagenDiseñoEmocional.jpg}
    \label{fig:my_label}
\end{figure}
\end{frame}


\section{Juegos}
\begin{frame}
\begin{itemize}
    \item Se han diseñado desde siempre para generar una experiencia de los mas divertida en el usuario
    \item "Los juegos son divertidos y atraen a jugadores no porque sean juegos, sino porque estan diseñados por gente que comprende qué es lo que hace que los jugadores se divierten y obtengan placer"
\end{itemize}
\begin{figure}
    \centering
    \includegraphics[width=7.0cm]{imagenJuego.jpg}
    \label{fig:my_label}
\end{figure}
\frametitle{El Juego}

\end{frame}


\begin{frame}
\frametitle{Circulo Magico}
\begin{itemize}
    \item Todo lo que se da en el juego queda en el juego.
    \item Es un espacio separado. ficticio, reglado por normas especiales y ajeno a nuestra realidad.
    \item "El juego no tiene efecto en la vida real"
\end{itemize}
\end{frame}


\begin{frame}
\frametitle{Esto no es del todo cierto ...}
\begin{itemize}
    \item Nostros podemos llevar situaciones del mundo real al juego...haciendolo mas divertido.
    \item Esto puede ayudarnos a llevar algunas situaciones un tanto tediosas.
\end{itemize}
\begin{figure}[h!]
    \centering
    \includegraphics[width=6.0cm]{imagenJuegoNiños.jpg}
    \label{fig:my_label}
\end{figure}
\end{frame}


\begin{frame}
\frametitle{¿A donde vamos con todo esto?}
Los juegos son un importante caso de exito como:
\begin{itemize}
    \item La creacion de un "engagement".
    \item Una buena interaccion con el usuario.
    \item Altos niveles de motivacion.
\end{itemize}
\begin{figure}
    \centering
    \includegraphics[width=7.0cm]{imagenExperiencia.png}
    \label{fig:my_label}
\end{figure}
\end{frame}

\section{Gamificación}
\begin{frame}
\frametitle{Concepto}
\begin{itemize}
    \item Es la aplicación de los elementos propios de los juegos.
    \item Tiene el fin de conseguir que las personas adopten ciertos comportamientos deseados.
    \item También se puede gamificar un producto, un servicio o aplicación.
\end{itemize}
\end{frame}

\begin{frame}
\frametitle{Ejemplos}
\begin{itemize}
    \item {\bf Socrative} Permite crear salones virtuales para trabajar las preguntas y claves de acceso a los estudiantes.
    \begin{figure}
    \includegraphics[width=4.0cm]{003f89b3747f076f93461b16c97e7ff1.jpg}
    \label{fig:my_label}
\end{figure}
    \item {\bf Kahoot!} Aplicación que permite generar juegos, desde estilo trivia hasta más complejos.
    \begin{figure}
    \includegraphics[width=4.0cm]{kahoot.png}
    \label{fig:my_label}
\end{figure}
\end{itemize}
\end{frame}

\begin{frame}
\frametitle{Elementos de juego}
\begin{itemize}
    \item {\bf Dinámica:} Experiencias que el diseñador quiere incorporar al juego, como la progresión.
    \item {\bf Mecánicas:} Que se necesita para que el diseñador pueda obtener las experiencias, como son los desafíos, logros o recompensas.
    \item {\bf Componentes:} Están implementados en las partes mecánicas del elemento del juego, como son los emblemas, niveles, misiones.
\end{itemize}
\end{frame}

\begin{frame}
\frametitle{Tipos de Jugadores}
\begin{itemize}
    \item  Ambiciosos
    \item Triunfadores
    \item Sociables
    \item Exploradores
\end{itemize}
\begin{figure}
    \centering
    \includegraphics[width=7.0cm]{imagen_2020-10-05_225833.png}
    \label{fig:my_label}
\end{figure}
\end{frame}

\begin{frame}
\frametitle{¿Qué nos motiva a jugar?}
La motivación es un impulso que nos permite tener cierta continuidad en la actividad que estamos realizando.\\
\begin{itemize}
    \item {\bf Morivación intrínseca: } La que proviene de nuestro interior. No necesitamos motivaciones extras.
    \item {\bf Motivación extrínseca: } Es la que esta asociada con un objetivo final.
    \item {\bf Motivación afectiva: } Está relacionada a las emociones.
\end{itemize}
\end{frame}


\begin{frame}
\frametitle{Teoria de Flujo}
\begin{itemize}
    \item Un estado mental o una experiecia "tan satisfactoria" que no queremos salir de ella incluso aunque sea peligroso o dificil.
    \item Sorpresa continua.
    \item Equilibrio entre habilidad y desafio.
    \begin{figure}
        \includegraphics[width=5.0cm]{usuario.jpg}
        \label{fig:my_label}
    \end{figure}
\end{itemize}
\end{frame}



\begin{frame}
\frametitle{Niveles}
\begin{itemize}
    \item Fantasia y curiosidad.
    \item  Motivar a los jugadores y seguir jugando.
    \item Sensasion de progreso.
    \begin{figure}
        \includegraphics[width=6.0cm]{niveles.jpg}
        \label{fig:my_label}
    \end{figure}
\end{itemize}
\end{frame}



\begin{frame}
\frametitle{Ciclos de Accion y Realimentacion}
\begin{itemize}
    \item Dentro del juego el usuario 
    tiene una serie de acciones a realizar.
    \begin{figure}
        \includegraphics[width=8.5cm]{Ciclo.jpg}
        \label{fig:my_label}
    \end{figure}
\end{itemize}
\end{frame}



\begin{frame}
\frametitle{Sistemas de puntuación y recompensa}
\begin{itemize}
    \item Es una forma facil de introducir gamificacion
    \item Actividades con sistemas PBL
    \item Una gamificacion intrinseca
    \begin{figure}
        \includegraphics[width=6.0cm]{PBL.jpg}
        \label{fig:my_label}
    \end{figure}
\end{itemize}
\end{frame}



\begin{frame}
\frametitle{Sensacion de progreso}
\begin{itemize}
    \item Indicadores de progreso y medallas o insignias.
    \item Indicadores de progresos tematicos.
    \begin{figure}
        \includegraphics[width=5.0cm]{premio.jpg}
        \label{fig:my_label}
    \end{figure}
\end{itemize}
\end{frame}



\begin{frame}
\frametitle{Recompensas virtuales o reales}
\begin{itemize}
    \item Tiene una gran importancia ya que motiva al jugador.
   
    \begin{figure}
        \includegraphics[width=6.0cm]{recompensa.jpg}
        \label{fig:my_label}
    \end{figure}
\end{itemize}
\end{frame}



\begin{frame}
\frametitle{La realimentacion}
\begin{itemize}
    \item Positiva.
    \item Divertida
    \item Adecuada.
    \begin{figure}
        \includegraphics[width=5.0cm]{realimentacion.jpg}
        \label{fig:my_label}
    \end{figure}
\end{itemize}
\end{frame}



\begin{frame}
\frametitle{La historia y narrativa}
\begin{itemize}
    \item Contar historias atrae la atencion da sentido y genera una grn experiencia.
    \item Realizar un viaje es una gran aventura.
    \item Viaje del heroe.
   
    \begin{figure}
        \includegraphics[width=5.0cm]{Historia.jpg}
        \label{fig:my_label}
    \end{figure}
\end{itemize}
\end{frame}



\begin{frame}
\frametitle{Onboarding}
\begin{itemize}
    \item Atraer al Usuario
    \item el usuario conocera lo que debe hacer la aplicacion 
     \begin{figure}
        \includegraphics[width=8.0cm]{onboarding.png}
        \label{fig:my_label}
    \end{figure}
\end{itemize}
\end{frame}




\begin{frame}
\frametitle{Inmerse}
\begin{itemize}
    \item debemos usar funciones progresivas 
    \item Uso de tutoriales
    \item Mantener entretenido al usuario
     \begin{figure}
        \includegraphics[width=5.0cm]{inmerse.jpg}
        \label{fig:my_label}
    \end{figure}
\end{itemize}
\end{frame}



\begin{frame}
\frametitle{Diseño basado en mapas de experiencia}
\begin{itemize}
    \item se involucra con la narrativa
     \begin{figure}
        \includegraphics[width=8.0cm]{experiencia.png}
        \label{fig:my_label}
    \end{figure}
\end{itemize}
\end{frame}




\begin{frame}
\frametitle{User Engagement}
\begin{itemize}
    \item Atrapar a los usuarios
    \item que vuelvan,que nos recuerden,que nos recomienden
    \begin{figure}
        \includegraphics[width=9.0cm]{retencion.png}
        \label{fig:my_label}
    \end{figure}
\end{itemize}
\end{frame}



\begin{frame}
\frametitle{Retencion}
\begin{itemize}
    \item Saber como retener a los usuarios
     \begin{figure}
        \includegraphics[width=9.0cm]{retencion.png}
        \label{fig:my_label}
    \end{figure}
\end{itemize}
\end{frame}



\begin{frame}
\frametitle{Incluir restricciones}
\begin{itemize}
    \item Las restricciones suelen generar motivacion al jugador 
    \item Las personas queremos lo que no tenemos 
    \item El usuario solo puede hacer algo en la aplicacion solo por hoy
    \item dar funcionabilidad a determinados usuarios
\end{itemize}
\end{frame}



\begin{frame}
\frametitle{Libertad}
\begin{itemize}
    \item Dar control al usuario sobre donde estamos y a donde queremos ir 
    \begin{figure}
        \includegraphics[width=10.0cm]{libertad.jpg}
        \label{fig:my_label}
    \end{figure}
\end{itemize}
\end{frame}



\begin{frame}
\frametitle{Emociones}
\begin{itemize}
    \item Añadir elementos que generen emociones (Sorpresa, Curiosidad, Entusiasmo, Admiración, Diversión, Frustración, Enojo, Placer, Satisfacción).
    \begin{figure}
        \includegraphics[width=8.0cm]{emociones.jpg}
        \label{fig:my_label}
    \end{figure}
    
\end{itemize}
\end{frame} 


\begin{frame}
\frametitle{Diseño de gamificacion}
\begin{itemize}
    \begin{figure}
        \includegraphics[width=10cm]{gamificacion.png}
        \label{fig:my_label}
    \end{figure}
\end{itemize}
\end{frame}



\begin{frame}
\frametitle{Contenido}
\begin{itemize}
    \item No se puede aumentar el valor intrinseco mediante la adicion de mecanicas de juego.
\end{itemize}
\end{frame}

\begin{frame}
\frametitle{Propuesta de aplicativo}
\begin{itemize}
    \item Implementar un juego persuasivo para ingresantes del ultimo año.
\end{itemize}
\begin{figure}
        \includegraphics[width=8.0cm]{capus.jpg}
        \label{fig:my_label}
    \end{figure}
\end{frame}



\section{References}
\begin{frame}
\frametitle{References}
\begin{itemize}
\item \href{https://www.youtube.com/watch?v=v23kdMadKQA}{Técnicas de Gamificación y mejora de la UX}
\item \href {https://youtu.be/evSQ3Lzw6eg}{Video del Aplicativo }
\item \href {https://www.educaciontrespuntocero.com/noticias/gamificacion-que-es-objetivos/}{Gamificación}
\item \href {https://docs.google.com/presentation/d/1kithbacPQYbR3hcHVanO_Qqp17kC4xdvGiT8p82FSwE/edit#slide=id.p}{Herramientas de Gamificación}
\end{itemize}
\end{frame}

\end{document}